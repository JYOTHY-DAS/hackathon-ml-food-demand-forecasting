\documentclass[12pt]{article}
\usepackage[utf8]{inputenc}
\usepackage{graphicx}
\usepackage[colorlinks=true, linkcolor=black]{hyperref}
\usepackage{float} 
\usepackage{caption}
\usepackage{subcaption}
\usepackage{multicol}
\usepackage{comment}
\usepackage{booktabs} % For better-looking tables 
\usepackage{array} % For more control over table column formatting
\usepackage{colortbl} %Coloring table
\usepackage{xcolor}%Support hex values of color

\begin{document}

%%%%%%%%%%%%%%%%%%%%%%%%%%%%%%%%%%%%%%%%%%%%%%%%%%%%%%%%%%%%
% Front page
%%%%%%%%%%%%%%%%%%%%%%%%%%%%%%%%%%%%%%%%%%%%%%%%%%%%%%%%%%%%
\title{Food demand forecasting}

\author{ Jyothy Das}
\date{21/02/2025}
\maketitle

\vspace{30pt}
\begin{center}
A project report on\\
Genpact machine learning hackathon project\\
'Food demand forecasting'.\\
\end{center}

\vspace{20pt}


%%%%%%%%%%%%%%%%%%%%%%%%%%%%%%%%%%%%%%%%%%%%%%%%%%%%%%%%%%%%
% Table of Contents, List of Figures, and List of Tables
%%%%%%%%%%%%%%%%%%%%%%%%%%%%%%%%%%%%%%%%%%%%%%%%%%%%%%%%%%%%
\tableofcontents
\listoffigures
\listoftables

%%%%%%%%%%%%%%%%%%%%%%%%%%%%%%%%%%%%%%%%%%%%%%%%%%%%%%%%%%%%
% Abstract
%%%%%%%%%%%%%%%%%%%%%%%%%%%%%%%%%%%%%%%%%%%%%%%%%%%%%%%%%%%%
\newpage
\section{Abstract}
Accurate demand forecasting is critical for meal delivery companies to optimize procurement and staffing decisions. This project aims to predict the demand for meals across multiple fulfillment centers for the next 10 weeks using historical order data, meal characteristics, and center-specific attributes. 
%%%%%%%%%%%%%%%%%%%%%%%%%%%%%%%%%%%%%%%%%%%%%%%%%%%%%%%%%%%%
% Problem Definition
%%%%%%%%%%%%%%%%%%%%%%%%%%%%%%%%%%%%%%%%%%%%%%%%%%%%%%%%%%%%
\section{Problem Definition}
\subsection{Overview }

\subsection{Problem statement}

%%%%%%%%%%%%%%%%%%%%%%%%%%%%%%%%%%%%%%%%%%%%%%%%%%%%%%%%%%%%
% Introduction
%%%%%%%%%%%%%%%%%%%%%%%%%%%%%%%%%%%%%%%%%%%%%%%%%%%%%%%%%%%%
\section{Introduction}

%%%%%%%%%%%%%%%%%%%%%%%%%%%%%%%%%%%%%%%%%%%%%%%%%%%%%%%%%%%%
% Literature Survey
%%%%%%%%%%%%%%%%%%%%%%%%%%%%%%%%%%%%%%%%%%%%%%%%%%%%%%%%%%%%
\section{Literature Survey}


\subsection{Key Areas of Study}

\subsection{Key Findings and Gaps}

\subsection{Relevance to Current Study}

%%%%%%%%%%%%%%%%%%%%%%%%%%%%%%%%%%%%%%%%%%%%%%%%%%%%%%%%%%%%
% Methodology
%%%%%%%%%%%%%%%%%%%%%%%%%%%%%%%%%%%%%%%%%%%%%%%%%%%%%%%%%%%%
\section{Methodology}
\subsection{Data Collection}

\subsubsection{Data Overview}

%%%%%%%%%%%%%%%%%%%%%%   Table 1 
\begin{center}
\begin{tabular}{||c|c||}
\hline
\hline
\rowcolor[HTML]{89A8B2} Column 1 & Column 2 \\
\hline
\hline
\rowcolor[HTML]{F8FAFC} Data 1 & Data 2 \\ % Light gray
\hline

\rowcolor[HTML]{D8EFD3} Data 4 & Data 5 \\
\hline

\rowcolor[HTML]{F8FAFC} Data 6 & Data 7 \\ % Light blue
\hline
\hline
\end{tabular}
\end{center}
\bigskip % Add some vertical space

 %%%%%%%%%%%%%%%%%%%%%%%%%%%%%%%%% Table 2
\begin{center}
\begin{tabular}{||c|c||}
\hline
\hline
\rowcolor[HTML]{89A8B2} Column 1 & Column 2 \\
\hline
\hline
\rowcolor[HTML]{F8FAFC} Data 1 & Data 2 \\ % Light gray
\hline

\rowcolor[HTML]{D8EFD3} Data 4 & Data 5 \\
\hline

\rowcolor[HTML]{F8FAFC} Data 6 & Data 7 \\ % Light blue
\hline
\hline
\end{tabular}
\end{center}
\bigskip % Add some vertical space
 %%%%%%%%%%%%%%%%%%%%%%%%%%%%%%%%% Table 3
\begin{center}
\begin{tabular}{||c|c||}
\hline
\hline
\rowcolor[HTML]{89A8B2} Column 1 & Column 2 \\
\hline
\hline
\rowcolor[HTML]{F8FAFC} Data 1 & Data 2 \\ % Light gray
\hline

\rowcolor[HTML]{D8EFD3} Data 4 & Data 5 \\
\hline

\rowcolor[HTML]{F8FAFC} Data 6 & Data 7 \\ % Light blue
\hline
\hline
\end{tabular}
\end{center}
\bigskip % Add some vertical space
%%%%%%%%%%%%%%%%%%%%%%


\textbf{Non-null Counts: }

\textbf{Missing Values: }There are no missing values, as all 29,999 rows are complete.
\vspace{5pt}

%%%%%%%%%%%%%%%%%%%%%
\subsection{Data Transformation}

\subsubsection{Data Synchronization: } Efforts have been made to synchronize the state and district names throughout the dataset, ensuring a uniform format.
\subsubsection{Column Removal}
\begin{comment}
As part of the data preprocessing, some columns were renamed for clarity and consistency, while irrelevant or redundant features were removed to streamline the dataset for analysis.\\

\vspace{5pt}
To streamline the dataset and focus on relevant variables for the analysis, several columns were removed, including those related to administrative IDs, season names, and insurance company details. This transformation resulted in a reduced set of features that better align with the goals of the project. \\

\vspace{5pt}

The following feature columns were dropped: 
\begin{multicols}{2}
\begin{itemize}
\item{sssyName.seasonName}
\item{sssyName.schemeName} 
\item{seasonID}
\item{schemeID}
\item{schemeCode}
\item{level3Name}
\item{stateID}
\item{stateCode}
\item{level3ID}
\item{level3}
\item{level3}
\item{Code}
\item{cropName}
\item{cropID}
\item{cropCode}
\item{pickingType}
\item{sssyID}
\item{year}
\item{policyStartDate}
\item{policyEndDate}
\item{isOfflineChallan}
\item{goiOfflineChallan}
\item{stateOfflineChallan}
\item{yieldEndDate}
\item{currentTime} 
\item{default}
\item{insuranceCompanyName}
\item{cutOfDate}
\item{tollFreeNumber}
\item{headQuaterAddress}
\item{headQuaterEmail}
\item{websiteLink}
\item{insuranceCompany.insuranceCompanyCode}
\item{insuranceCompany.insuranceCompanyID}
\item{isOpen}
\item{cnStarted}
\item{unit}
\item{ayTy}
\item{Scheme}
\item{Start}
\end{itemize}
\end{multicols}
\vspace{5pt}
\end{comment}
\subsubsection{Column Rename}
To enhance clarity and improve readability, several columns were renamed to more intuitive and consistent names. 
\vspace{5pt}
\subsubsection{}


\subsubsection{}


\subsubsection{}


\subsubsection{}

%%%%%%%%%%%%%%%%%%%%%%%%%%%%%%%%%%%%%%%%%%%%%%%%%%%%%%%%%%%%
							%Model development
%%%%%%%%%%%%%%%%%%%%%%%%%%%%%%%%%%%%%%%%%%%%%%%%%%%%%%%%%%%%
\subsection{Model Development}
Insert Model Development details here.

%%%%%%%%%%%%%%%%%%%%%%%%%%%%%%%%%%%%%%%%%%%%%%%%%%%%%%%%%%%%
							% Results and Discussion
%%%%%%%%%%%%%%%%%%%%%%%%%%%%%%%%%%%%%%%%%%%%%%%%%%%%%%%%%%%%
\section{Results and Discussion}
Insert Results here.

%%%%%%%%%%%%%%%%%%%%%%%%%%%%%%%%%%%%%%%%%%%%%%%%%%%%%%%%%%%%
							% Conclusion
%%%%%%%%%%%%%%%%%%%%%%%%%%%%%%%%%%%%%%%%%%%%%%%%%%%%%%%%%%%%
\section{Conclusion}
Insert Conclusion here.

%%%%%%%%%%%%%%%%%%%%%%%%%%%%%%%%%%%%%%%%%%%%%%%%%%%%%%%%%%%%
% References
%%%%%%%%%%%%%%%%%%%%%%%%%%%%%%%%%%%%%%%%%%%%%%%%%%%%%%%%%%%%
\section{References}
Insert References here.

\end{document}
